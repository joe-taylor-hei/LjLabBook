\section{Machine Learning}

\begin{itemize}
\item
How do you want to frame the problem?\newline
Regression $\rightarrow$ target value?
Classification $\rightarrow$ target value?\newline
What features do you hypothesize to be good predictors?

\item
Setup workspace and version control.\newline
Track all SQL and python (inc. jupyter notebook) files.

\item
Write a SQL query to get the desired dataset(s).
\textit{It would be neater as a single SQL file.}\newline
Save the dataset in a time-stamped directory.
Copy the SQL file here as a reference.

\item
\textbf{Jupyter notebook one (loading and exploration)}\newline
\textit{Tip: try to keep notebooks neat, documented, and written in fairly big blocks.}\newline
\textit{Tip: use a version index to keep tabs of iterations.}\newline
Load data.\newline
Drop unwanted columns.\newline
Perform manipulations (e.g. merging tables and dropping duplicates).\newline
Clean data (e.g. deal with null entries).\newline
\textit{Perform lots of sanity checks along the way.}\newline
\newline
Divide dataset into training and test sets.\newline
Create a copy of the training set and use to gain insights. You could look at:\newline
- Distributions of predictors (you might need to transform predictors).\newline
- Distributions and correlations between predictors and targets.\newline
- Predictor combinations.\newline
\textit{You might now want to add new features or refine existing features and, thus, iterate.}

\item
\textbf{Jupyter notebook two (training and evaluation)}\newline
\textit{Use the first notebook to neatly create the training and test sets.}\newline
Create a copy of the training set predictors.\newline
Create a separate copy of the training set targets.\newline
\newline
Create and apply a pipeline to the training set predictors:\newline
- Encode categorical attributes.\newline
- Feature scale numerical attributes.\newline
- Apply custom transformations.\newline
\newline
Train some different models. For each model:\newline
- Train and evaluate.\newline
- Cross-validate.\newline
- Fine tune model hyperparameters (then repeat the above steps).\newline
- Analyze final model (e.g. study the importance of each feature for making predictions).\newline
\newline
Evaluate final model on the test set.
\end{itemize}

\newpage