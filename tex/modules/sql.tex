\section{SQL}

Creating Tables:\newline
\texttt{DROP TABLE IF EXISTS lut;}\newline
\texttt{CREATE TEMP TABLE lut (code TEXT, full\char`_string TEXT);}\newline
\texttt{INSERT INTO lut VALUES (\textquotesingle inc\textquotesingle, \textquotesingle Included\textquotesingle);}\newline
\texttt{INSERT INTO lut VALUES (\textquotesingle exc\textquotesingle, \textquotesingle Excluded\textquotesingle);}\newline
\texttt{INSERT INTO lut VALUES (\textquotesingle n/a\textquotesingle, \textquotesingle Unspecified\textquotesingle);}

\vspace{\baselineskip}
Manipulating Tables:\newline
\texttt{DELETE FROM lut WHERE code = \textquotesingle n/a\textquotesingle;}\newline
\texttt{ALTER TABLE lut ADD id INT;}\newline
\texttt{UPDATE lut SET id = 1 WHERE code = \textquotesingle inc\textquotesingle;}\newline
\texttt{UPDATE lut SET id = 2 WHERE code = \textquotesingle exc\textquotesingle;}

\vspace{\baselineskip}
Querying Tables:\newline
\texttt{SELECT}\newline
\texttt{*}\newline
\texttt{, weight / (height * height) AS bmi}\newline
\texttt{FROM customers}\newline
\texttt{WHERE has\char`_gp = TRUE}\newline
\texttt{AND weight > 75.0}\newline
\texttt{AND height IS NOT NULL}\newline
\texttt{AND age BETWEEN 16 AND 30}\newline
\texttt{AND eye\char`_colour IN (\textquotesingle blue\textquotesingle, \textquotesingle grey\textquotesingle)}\newline
\texttt{AND name LIKE \textquotesingle B\%\textquotesingle}\newline
\texttt{ORDER BY name}\newline
\texttt{LIMIT 1000}

\vspace{\baselineskip}
Combining Data:\newline
\texttt{SELECT * FROM customers}\newline
\texttt{UNION ALL}\newline
\texttt{SELECT * FROM employees}

\vspace{\baselineskip}
Aggregating Data:\newline
\texttt{SELECT}\newline
\texttt{name}\newline
\texttt{, COUNT(*) AS freq}\newline
\texttt{, MAX(age) AS max\char`_age}\newline
\texttt{, SUM(weight) AS total\char`_weight}\newline
\texttt{, SUM(CASE WHEN age = 18 THEN 1 ELSE 0 END) AS freq\char`_age\char`_18}\newline
\texttt{FROM customers}\newline
\texttt{GROUP BY 1}\newline
\texttt{ORDER BY 1}

\newpage
`IF' Statements:\newline
\texttt{SELECT}\newline
\texttt{>~~~COALESCE(eye\char`_colour, \textquotesingle unknown\textquotesingle) AS eye\char`_colour}\newline
\texttt{>~~~, CASE}\newline
\texttt{>~~~~~~~WHEN age > 65 THEN \textquotesingle OAP\textquotesingle}\newline
\texttt{>~~~~~~~WHEN age > 17 THEN \textquotesingle Adult\textquotesingle}\newline
\texttt{>~~~~~~~ELSE \textquotesingle Child\textquotesingle}\newline
\texttt{>~~~END AS age\char`_cat}\newline
\texttt{FROM customers}

\vspace{\baselineskip}
Inner Join:\newline
\textit{Returns records that have matching values in both tables.}\newline
\texttt{SELECT}\newline
\texttt{clk.*}\newline
\texttt{, pr.provider\char`_name}\newline
\texttt{FROM holiday.fct\char`_click clk}\newline
\texttt{INNER JOIN common.dim\char`_provider pr USING (provider\char`_id)}

\vspace{\baselineskip}
Left Join:\newline
\textit{Returns \underline{all} records from the left table (first listed), and the matched records from the right table.}\newline
\texttt{SELECT}\newline
\texttt{*}\newline
\texttt{FROM holiday.fct\char`_click clk}\newline
\texttt{LEFT JOIN common.dim\char`_provider pr ON clk.provider\char`_id = pr.id}

\vspace{\baselineskip}
Full Join:\newline
\textit{Returns \underline{all} records from the left and right tables.}\newline
\texttt{SELECT}\newline
\texttt{COALESCE(clk.date, bk.date) AS date}\newline
\texttt{, COALESCE(clk.revenue, 0) + COALESCE(bk.revenue, 0) AS revenue}\newline
\texttt{FROM click\char`_revenue clk}\newline
\texttt{FULL OUTER JOIN booking\char`_revenue bk USING (date)}



The WITH clause is an optional prefix for SELECT.
You can then use the results in the main SELECT statement:
\begin{easylist}[itemize]
\ListProperties(Style*=$\bullet$ , FinalMark={)})
& \texttt{WITH query\char`_name (column\char`_name1, ...) AS\newline
>~~~(SELECT ...)\newline
SELECT ...}
\end{easylist}

\underline{Window Functions:}\newline
They perform calculations over a set of rows.
They are indicated by the OVER clause.\newline
\newline
Here is what it all means:\newline
OVER = using all rows\newline
PARTITION BY = with the same...\newline
ORDER BY = applying function sequentially ordered by...
% 
\begin{easylist}[itemize]
\ListProperties(Style*=$\bullet$ , FinalMark={)})
& \texttt{SELECT firstname, lastname, date\char`_started,\newline
>~~~COUNT(*) OVER (PARTITION BY year(date\char`_started)) AS NumPerYear\newline
>~~~FROM Employee;}
\end{easylist}

\newpage