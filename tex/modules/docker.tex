\section{DOCKER}

You can think of a computer operating system as having two parts:
\vspace{-3.5mm}
\begin{itemize}
\item
\textbf{The Kernel:} talks to the hardware and the network. In general we don't touch this.\newline
The kernel is accessed via APIs from \textit{Userland.}
\vspace{-1.0mm}
\item
\textbf{Userland:} is where your programs/applications live. We do touch this!
\end{itemize}

Application Problems:
\vspace{-3.5mm}
\begin{itemize}
\item
Running distinct applications in Userland if they have different library dependencies.
\vspace{-1.0mm}
\item
Running the same application, but on a different computer,
if the Userlands have different library package versions.
\end{itemize}

These problems can be solved using Virtual Machines or Containers:
\vspace{-3.5mm}
\begin{itemize}
\item
\textbf{VMs} represent the Userland and Kernel of an OS required for an application.
\vspace{-1.0mm}
\item
\textbf{Containers} represent just the Userland of an OS required for an application.\newline
In Docker, the containers running share the host OS kernel, which is more efficient.
\end{itemize}

\textbf{Images} are essentially the instructions of how to build a Container (or VM).\newline
- The Container is the thing you get when you build an Image.\newline
- For example, you can instantiate multiple Docker Containers from the same Docker Image.\newline
- Docker images are the thing that you would push/pull.\newline
- Images can be built as layers on top of other images (you don't build them from scratch).

Commands:
\begin{easylist}[itemize]
\ListProperties(Style*=$\bullet$ , FinalMark={)}) % FinalMark indicates the end of the list properties and must always be used

& \texttt{docker images} -- prints information about all your docker images.

& \texttt{docker pull supertest2014/nyan:latest} \newline -- download a docker image (if not stated, the `latest' tag is assumed).

& \texttt{docker run supertest2014/nyan} or \texttt{<IMAGE-ID>}
\newline -- creates a container instance of an image and runs it.
\newline -- if you don't have the image it will download it (default from DockerHub).
\newline -- it's like you've built the computer \textit{and} turned it on.

& \texttt{docker rmi supertest2014/nyan} or \texttt{<IMAGE-ID>} -- delete a docker image.

& \texttt{docker ps} -- list the docker containers that are running.

& \texttt{docker stop <CONTAINER-ID>} or \texttt{<NAME>} -- stop a docker container running.

& \texttt{docker ps -a} -- list \textit{all} the containers.
\newline -- stopped containers are still tracked (if you turn off a computer, you still have it).

& \texttt{docker start <CONTAINER-ID>} or \texttt{<NAME>} -- restart a docker container.

& \texttt{docker rm <CONTAINER-ID>} or \texttt{<NAME>} -- delete a docker container.

\end{easylist}

A Dockerfile is a text document that contains all the commands required to assemble an image:\newline
\texttt{docker build --tag python-app .}

\newpage