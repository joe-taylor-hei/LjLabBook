\section{PYTHON}

Basics:
\begin{easylist}[itemize]
\ListProperties(Style*=$\bullet$ , FinalMark={)})
& \texttt{python} -- start interactive python shell.
& \texttt{ipython} -- start interactive ipython shell.
\newline You can tab complete and use bash commands e.g. \texttt{!pwd}.
& \texttt{CNTRL D} -- exit interactive python shell.
& \texttt{python <file.py>} -- execute python script.
\end{easylist}

\vspace{\baselineskip}
\textbf{JUPITER NOTEBOOKS}\newline
Jupiter notebooks run on your browser.
They are like ipython sessions that can be saved.
\newline You can also edit previous lines.
\begin{easylist}[itemize]
\ListProperties(Style*=$\bullet$ , FinalMark={)})
& \texttt{jupyter notebook} -- to set it all up.
\newline Then start a new notebook or select an old \texttt{.ipynb} file.
& \texttt{SHFT ENTER} -- run cell and select next.
\end{easylist}

\vspace{\baselineskip}
\textbf{CONDA}
is a package and environment management system (first download ANACONDA):
\begin{easylist}[itemize]
\ListProperties(Style*=$\bullet$ , FinalMark={)})
& \texttt{conda info} -- verify conda is installed, check version number.
% 
& \texttt{conda env list} -- list all conda environments.
& \texttt{conda create --name NEWENV} -- create a new conda environment.
& \texttt{conda create --clone OLDENV --name NEWENV} -- clone a conda environment.
& \texttt{conda env remove --name NEWENV} -- delete a conda environment.
% 
& \texttt{conda activate NEWENV} -- activate a conda environment.
& \texttt{conda deactivate} -- deactivate the current conda environment.
% 
& \texttt{conda list} -- list all packages and versions (in the active environment).
& \texttt{conda list --revisions} -- list the history of each change to the environment.
& \texttt{conda install --revision N} -- restore environment to a previous version.
% 
& \texttt{conda install PACKAGENAME} -- install a new package.
& \texttt{conda update PACKAGENAME} -- update an installed package.
& \texttt{conda remove PACKAGENAME} -- remove a package.
& \texttt{pip install PACKAGENAME} -- install a package directly from PyPI (into conda env).
\end{easylist}

\vspace{\baselineskip}
\textbf{VIRTUAL ENV}
\begin{easylist}[itemize]
\ListProperties(Style*=$\bullet$ , FinalMark={)})
& \texttt{python -m venv venv}
& \texttt{source venv/bin/activate}
& \texttt{pip install -r requirements.txt}
& \texttt{deactivate}
\end{easylist}

\newpage