\section{GITHUB}

Git is a version control system.
GitHub is the online store.

\vspace{\baselineskip}
GitHub -- i.e. online.
\begin{itemize}
\item
To setup a new repository online you must first create an empty repo on GitHub.
In your local repo you can then set this new GitHub repo as your remote.
% And then you're off.

\item
If you want to collaborate on an existing repo you should fork this repo to your account.
To get it locally you should then clone your GitHub version of the repo.

\item
If you want to merge your code with a collaborator on GitHub use the 'New Pull Request' button.
\end{itemize}


Git Setup.
\begin{easylist}[itemize]
\ListProperties(Style*=$\bullet$ , FinalMark={)}) % FinalMark indicates the end of the list properties and must always be used
& \texttt{git init [projectDir]} -- set up git. You will be on a branch called 'master'.
& \texttt{git clone git@github.com:joseph-taylor/LjLabBook.git <localRepoName>} -- clone a repo to local from GitHub. This will automatically set a remote called 'origin'.
& \texttt.gitignore -- text file telling git what files to ignore.
& \texttt{git config --global core.editor "emacs -nw"} -- set emacs as default git text editor.
\end{easylist}

Git Basics.
\begin{easylist}[itemize]
\ListProperties(Style*=$\bullet$ , FinalMark={)}) % FinalMark indicates the end of the list properties and must always be used
& \texttt{git status} -- prints the status of your branch.
& \texttt{git add <files>} -- stages changes to files.
& \texttt{git reset HEAD <files>} -- unstage files.
& \texttt{git rm <files>} -- remove files and have git stage the changes.
& \texttt{git mv <files> <location>} -- move files and have git stage the changes.
& \texttt{git commit [-m "message"]} -- commit the changes.
\end{easylist}

Git Commit Labels.
\begin{easylist}[itemize]
\ListProperties(Style*=$\bullet$ , FinalMark={)}) % FinalMark indicates the end of the list properties and must always be used
& Each commit is labelled by a commit hash. Use \texttt{git log} to find them.
& \texttt{HEAD} -- a special label referring to your current commit on your current branch.
& \texttt{commitLabel\^{}} -- refers to a NEW label, one commit before the label provided.
& \texttt{commitLabel\^{}\^{}} -- refers to a NEW label, two commits before the label provided.
& \texttt{commitLabel$\sim$N} -- refers to a NEW label, N commits before the label provided.
\end{easylist}

Git Useful.
\begin{easylist}[itemize]
\ListProperties(Style*=$\bullet$ , FinalMark={)}) % FinalMark indicates the end of the list properties and must always be used
& \texttt{git diff <file>} -- diff between file and version from last commit.
& \texttt{git diff <commitLabel> <file>} -- diff between file and version from the stated commit.
& \texttt{git diff <commitLabelOne> <commitLabelTwo> <file>} -- diff of file between two different commits.
& \texttt{git log} -- log of git commit messages and hashes.
& \texttt{git help <command>} -- git help pages.
\end{easylist}

Git Branching.
you get conflicts when changes are made to the same files.
\begin{easylist}[itemize]
\ListProperties(Style*=$\bullet$ , FinalMark={)}) % FinalMark indicates the end of the list properties and must always be used
& \texttt{git checkout -b <newBranchName>} -- create new branch and change to it.
& \texttt{git checkout <oldBranchName>} -- change to an existing branch.
& \texttt{git merge <branchNameToMerge> [-m "message"]} -- merge changes from stated branch into current branch.
& \texttt{git branch -d <branchNameToDelete>} -- delete branch (can only do if changes are merged).
\end{easylist}

Git Stashing.
If you want to change branch, but don't want to commit current work first.
\begin{easylist}[itemize]
\ListProperties(Style*=$\bullet$ , FinalMark={)}) % FinalMark indicates the end of the list properties and must always be used
& \texttt{git stash} -- saves your changes and goes back to the HEAD commit.
& \texttt{git stash list} -- look at your stored stashes.
& \texttt{git stash apply [stash@{N}]} -- apply a stash. If you have more than one stash you should name it.
& \texttt{git stash clear} -- remove all stashes.
& \texttt{git stash drop <stash@{N}>} -- remove a specific stash.
\end{easylist}

Reversing Changes in Git.
\begin{easylist}[itemize]
\ListProperties(Style*=$\bullet$ , FinalMark={)}) % FinalMark indicates the end of the list properties and must always be used
& \texttt{git checkout -- <file>} -- undo uncommited changes to a file (-- avoids conflict with branch names).
& \texttt{git checkout <commitLabel>} --  detached HEAD state: can view state of old commit but you are not on a branch. Branch off if you want to work from here.
& \texttt{git revert <commitLabel>} -- reverses \textbf{the changes of} the stated commit. This change happens in the form of a new commit. Just exit the emacs situation.
& \texttt{git reset <commitLabel>} -- moves a branch back in time to an older commit. The commits that were on top of it no longer exist. Not good when collaborating.
\end{easylist}

Git to GitHub.
\begin{easylist}[itemize]
\ListProperties(Style*=$\bullet$ , FinalMark={)}) % FinalMark indicates the end of the list properties and must always be used
& \texttt{git remote [-v]} -- print the GitHub remotes for the repo.
& \texttt{git remote add <remoteName> git@github.com:joseph-taylor/LjLabBook.git} -- create a new remote.
& \texttt{git push <remoteName> <branchName>} -- push commits on the local branch to the remote branch.
& \texttt{git pull <remoteName> <branchName>} -- pull commits on remote branch to the local branch.
\end{easylist}

\vspace{\baselineskip}
\vspace{\baselineskip}
Useful Webpages
\begin{itemize}
  \item \texttt{https://learngitbranching.js.org/} - practise git branching conceputally.
  \item \texttt{https://help.github.com/articles/connecting-to-github-with-ssh/} - Connecting to GitHub with SSH.
\end{itemize}

\newpage