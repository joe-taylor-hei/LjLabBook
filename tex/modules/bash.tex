\section{BASH}

\textit{Everything in UNIX is either a file or a process.
Processes can take additional options.}

Basics:
\begin{easylist}[itemize]
\ListProperties(Style*=$\bullet$ , FinalMark={)}) % FinalMark indicates the end of the list properties and must always be used
& \texttt{cd} -- change directory.
& \texttt{ls} -- list the files/directories.
& \texttt{pwd} -- print working directory.
& \texttt{mkdir} -- make directory.
& \texttt{touch} -- create file.
& \texttt{rm, rmdir} -- remove.
& \texttt{cp} -- copy
& \texttt{mv} -- move (rename)
& \texttt{echo} -- print to command line.
& \texttt{wc} -- word count.
\end{easylist}

Other:
\begin{easylist}[itemize]
\ListProperties(Style*=$\bullet$ , FinalMark={)}) % FinalMark indicates the end of the list properties and must always be used
& \texttt{file <file>} -- tells you what type of file it is.
& \texttt{tar} -- (un)compress files and directories. 
& \texttt{du -sh} -- check the disk usage.
& \texttt{df [path]} -- reports on amount of available disk space.
& \texttt{ps [ux[ww]]} -- gives you info abouts your processes. 
& \texttt{top} -- gives you info abouts the systems processes.
& \texttt{who} -- see who is on the system with you.
& \texttt{chmod X$\pm$Y} -- change permissions.\newline\textbf{X} = \texttt{u}:user, \texttt{g}:group, \texttt{o}:other, \texttt{a}:all. \textbf{Y} = \texttt{r}:read, \texttt{w}:write, \texttt{x}:execute.
& \texttt{<command> --help, man <command>} -- how to use a command.
& \texttt{say <quote>} -- voice output from the terminal.
& \texttt{open -a [application] filename} -- open a file.
% & \texttt{date} -- displays the date and time.
& \texttt{echo one two three | xargs mkdir} -- xargs allows tools to accept standard input as arguments.
\end{easylist}

Navigation:
\begin{easylist}[itemize]
\ListProperties(Style*=$\bullet$ , FinalMark={)}) % FinalMark indicates the end of the list properties and must always be used
& \texttt{/} -- root directory.
& \texttt{./} -- current directory, alias for \texttt{\$PWD}.
& \texttt{../} -- parent directory.
& $\sim$ -- home directory, alias for \texttt{\$HOME}.
& \texttt{-} -- previous directory, alias for \texttt{\$OLDPWD}.
& \texttt{dirs} -- directory stack. By default the first value is \texttt{\$PWD}. \verb!-c! cleans stack. \verb!-v! shows enumerated version; cd into them using \verb!cd! $\sim$\verb!n!.
& \texttt{pushd .} -- add current directory to stack. 
& \texttt{*} -- wildcards for file and directory names.
& \texttt{?} -- single character wildcards.
& \texttt{Tab} -- auto complete until an ambiguity arises.
& \texttt{Tab Tab} -- displays the ambiguities.
\end{easylist}

Text:
\begin{easylist}[itemize]
\ListProperties(Style*=$\bullet$ , FinalMark={)}) % FinalMark indicates the end of the list properties and must always be used
& \texttt{cat, head, tail} -- display text file (all, top, bottom).
& \texttt{more} -- scroll through text file (space bar for next page).
& \texttt{sort <stream/file>} -- sort text.
& \texttt{diff} -- differences between files.
& \texttt{command > file} -- redirect standard output to a file (use \verb!>&! to include errors).
& \texttt{command >> file} -- append standard output to a file.
& \texttt{command1 | command2} -- pipe output of command-1 to input of command-2.
\end{easylist}

History and Kill:
\begin{easylist}[itemize]
\ListProperties(Style*=$\bullet$ , FinalMark={)}) % FinalMark indicates the end of the list properties and must always be used
& $\uparrow$ and $\downarrow$ -- scroll command history.
& \texttt{history} -- show command history (use \texttt{grep} to filter).
& \texttt{CNTRL R} -- navigate bash history (press repeatedly to search further back).
& \texttt{clear} -- clears the screen.
& \texttt{CNTRL C} -- kill a foreground process in the foreground.
\newpage
& \texttt{CNTRL Z} -- put a foreground process in the background. Use \texttt{\&} at the end of a command to do this automatically.
& \texttt{bg} -- look at background processes, provides a job number.
& \texttt{fg \% <jobNumber>} -- put background process in foreground.
& \texttt{kill \% <jobNumber>} -- kill background process.
\end{easylist}

Very Useful, Can Never Remember:
\begin{easylist}[itemize]
\ListProperties(Style*=$\bullet$ , FinalMark={)}) % FinalMark indicates the end of the list properties and must always be used
& \texttt{sed} -- \textbf{s}tream \textbf{ed}itor. 
& \texttt{sed \textquotesingle s/<old>/<new>/g\textquotesingle~<stream/file>} -- sed for substitution.
\newline Use \verb!-i! for in-file replacement. Note you can use different delimiters.
& \texttt{find <directory> -name <file name>} -- find the path to a file.\newline Searches recursively by default. Other options exist e.g. file size.
& \texttt{grep <expression> <stream/file>} -- search text file for an expression.
\newline Option \verb!-i! ignores case; \verb!-r! searches recursively.
\end{easylist}

Secure Shell:
\begin{easylist}[itemize]
\ListProperties(Style*=$\bullet$ , FinalMark={)}) % FinalMark indicates the end of the list properties and must always be used
& \texttt{scp <local path> <user>@<hostname>:<destination path>} \newline-- secure copy.
& \texttt{ssh -XY <user>@<hostname>} -- secure shell, to log in to a remote server.
& \texttt{sshfs <user>@<hostname>:/path/to/base/ <locMountDir> -o volname=NAME} -- secure shell file system.
& \texttt{display <image file>} -- view things in ssh.
& \texttt{exit} -- exit ssh session.
\end{easylist}

Screen:
\begin{easylist}[itemize]
\ListProperties(Style*=$\bullet$ , FinalMark={)}) % FinalMark indicates the end of the list properties and must always be used
& \texttt{screen} -- start screen session (environments are not transferred over). 
& \texttt{Ctrl A} then \texttt{D} -- detach screen session. 
& \texttt{exit} -- kill screen session. 
& \texttt{screen -r <sessionID>} -- re-attach screen session. 
& \texttt{screen -ls} -- list screen sessions. 
& \texttt{screen -X -S <sessionID> kill} -- kill screen session without attaching it. 
\end{easylist}

\newpage
Environment Variables And Loops:
\begin{easylist}[itemize]
\ListProperties(Style*=$\bullet$ , FinalMark={)}) % FinalMark indicates the end of the list properties and must always be used
& \texttt{env} -- print environmental variables. \texttt{PATH} is a colon seperated list of dirs that your shell searches for an executable.
& \texttt{export PATH="/path/to/bin:\$PATH"} -- example of setting an environmental variable.
& \texttt{myvar=123abc} -- example of declaring a variable (string only).
& \texttt{echo \$myvar} -- example of using variable.
& \texttt{for myvar in *.pdf; do ls \$myvar; echo; done} \newline-- specific example of for loop.
& \texttt{while true; do <command1>; <command2>; etc; done} \newline-- run a command periodically.
& \texttt{alias <shortcut>="<regularExpression>"} -- alias commands.
& \texttt{source <shell file>} -- evaluates the file.
& \texttt{.bashrc} OR \texttt{.bash\char`_profile} -- for permanent aliases and setting of variables.
\end{easylist}

Useful Webpages:
\begin{easylist}[itemize]
\ListProperties(Style*=$\bullet$ , FinalMark={)}) % FinalMark indicates the end of the list properties and must always be used
& \texttt{https://ss64.com/bash/} -- all the commands!
\end{easylist}

\newpage